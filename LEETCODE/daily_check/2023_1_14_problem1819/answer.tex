\documentclass[12pt]{ctexart}

\usepackage{amsmath}
\usepackage{mathabx} %添加证毕符号
\usepackage{cases}

\begin{document}

\title{1819. 序列中不同最大公约数数目 解析}
\author{张翼翔\thanks{E-mail:21371055@buaa.edu.cn}}
\date{2023/1/14}

\maketitle

\section{方法一: 枚举}

\subsection{思路与算法}
题目要求找到所有非空子序列中不同的最大公约数的数目, 我们可以尝试枚举所有可能的最大公约数. 假设 $p$ 为一个序列 $A = [a_0, a_1, \cdots, a_k]$ 的最大公约数, 令 $a_i = ci \times p$, 则序列即为 $A = [c_0 \times p, c_1 \times p, c_2 \times p, \cdots, c_k times p]$, 根据最大公约数的性质可知此时 $gcd(a_0, a_1, a_2, \cdots,  a_k) = p$, 则可以推出 $gcd(c_0, c_1, c_2, \cdots, c_k) = 1$. 此时我们在序列 $A$ 中添加 $p$ 的任意倍数 $a_{k + 1} = c_{k + 1} \times p$ 时, 则序列 $A$ 的最大公约数依然为 $p$, 即此时 $gcd(a_0, a_1, a_2, \cdots, a_k, a_{k + 1}) = p$.

根据以上推论我们可以得出结论, 如果 $x$ 为数组 $nums$ 中的某个序列的最大公约数, 则数组中所有能够被 $x$ 整除的元素构成的最大公约数一定为 $x$. 这样的数我们也称之为 \textbf{基本的}. 存在以 $x$为最大公约数的充分必要条件就是:
\begin{equation}
    \text{对于} \forall m \forall x ((m \in [1, \max(nums)] \to m \equiv 0 (\bmod\ x)) \to (m \geqslant y)). \label{subsec:eq1}
\end{equation}

根据上述结论, 我们可以枚举所有可能的最大公约数 $x$, 其中 $x \in [1, \max(nums)]$, 然后对数组中所有可以整除 $x$ 的元素求最大公约数, 判断最后求出的最大公约数是否等于 $x$ 即可. 如果等于, 说明这些数恰好以 $x$ 为最大公约数, 则 $ans++$.否则还有比 $x$ 更大的公约数 $y$, $x$ 不是最大公约数.

\subsection{代码}
\begin{verbatim}
  class Solution {
    public:
        int countDifferentSubsequenceGCDs(vector<int>& nums) {
            int maxVal = 0;
            int ans = 0;
            for (vector<int>::iterator it
          = nums.begin(); it != nums.end(); ++it)
            {
                maxVal = max(maxVal, (*it));
            }
            vector<bool> occured(maxVal + 2, false);
            for (vector<int>::iterator it 
            = nums.begin(); it != nums.end(); ++it)
            {
                occured[(*it)] = true;
            }
            for (int i = 1; i <= maxVal; i++)
            {
                int gcdVal = 0;
                for (int k = i; k <= 
                maxVal; k = k + i)
                {
                    if (occured[k])
                    {
                        if (gcdVal == 0)
                        { 
                            gcdVal = k;
                        }
                        else
                        {
                            gcdVal = __gcd(gcdVal, k);
                        }
                    }
                }
                ans += (i == gcdVal);
            }
            return ans;
        }
        //O(n + maxVal log(maxVal))
    };
\end{verbatim}

\subsection{复杂度分析}
\subsubsection{时间复杂度}
$O(n + \max(nums)\log(\max(nums)))$, 其中 $n$ 表示数组的长度, $\max(nums)$ 表示数组中的最大元素. 我们首先需要遍历一遍数组, 然后从 $1$ 到 $\max(nums)$ 一次枚举每个可能的最大公约数(公式~\eqref{subsec:eq1}). 对于给定的数 $x$, 每次时间复杂度为 $\displaystyle \frac{\max(nums)}{x}$.

而 $\displaystyle n + \frac{n}{2} + \frac{n}{3} + \frac{n}{3} + \cdots + 1 = \sum_{i = 1}^{n} \frac{n}{i} \approx n\log(n)$, 因此在枚举的时候需要的时间为 $\max(nums)\log(\max(nums))$, 总时间复杂度 $O(n + \max(nums)\log(\max(nums)))$.

\subsubsection{空间复杂度}
$O(\max(nums))$ , 其中 $\max(nums)$ 表示数组中的最大元素. 我们需要一个 $O(\max(nums))$ 空间来标记数组中的每个元素是否出现过($[1, \max(nums)]$ 对应元素).
\end{document}