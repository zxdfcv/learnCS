方法一:直接模拟
思路与算法

题目要求,一步操作中,对于每个索引 ii,变换规则如下:

如果 ii 为偶数,那么 \textit{arr}[i] = \textit{perm}[\dfrac{i}{2}]arr[i]=perm[ 
2
i
​
 ];
如果 ii 为奇数,那么 \textit{arr}[i] = \textit{perm}[\dfrac{n}{2} + \dfrac{i-1}{2}]arr[i]=perm[ 
2
n
​
 + 
2
i−1
​
 ];
然后将 \textit{arr}arr 赋值给 \textit{perm}perm。

我们假设初始序列 \textit{perm} = [0,1,2,\cdots,n-1]perm=[0,1,2,⋯,n−1],按照题目上述要求的变换规则进行模拟,直到 \textit{perm}perm 重新变回为序列 [0,1,2,\cdots,n-1][0,1,2,⋯,n−1] 为止。每次将 \textit{perm}perm 按照上述规则变化产生数组 \textit{arr}arr,并将 \textit{arr}arr 赋给 \textit{perm}perm,然后我们检测 \textit{perm}perm 是否回到原始状态并计数,如果回到原始状态则中止变换,否则继续变换。

代码

Python3C++JavaC#CJavaScriptGolang

class Solution:
    def reinitializePermutation(self, n: int) -> int:
        perm = list(range(n))
        target = perm.copy()
        step = 0
        while True:
            step += 1
            perm = [perm[n // 2 + (i - 1) // 2] if i % 2 else perm[i // 2] for i in range(n)]
            if perm == target:
                return step
复杂度分析

时间复杂度:O(n^2)O(n 
2
 ),其中 nn 表示给定的元素。根据方法二的推论可以知道最多需要经过 nn 次变换即可回到初始状态,每次变换需要的时间复杂度为 O(n)O(n),因此总的时间复杂度为 O(n^2)O(n 
2
 )。

空间复杂度:O(n)O(n),其中 nn 表示给定的元素。我们需要存储每次变换中的过程变量,需要的空间为 O(n)O(n)。

方法二:数学
思路与算法

我们需要观察一下原排列中对应的索引变换关系。对于原排列中第 ii 个元素,设 g(i)g(i) 为进行一次操作后,该元素的新的下标,题目转化规则如下:

如果 g(i)g(i) 为偶数,那么 \textit{arr}[g(i)] = \textit{perm}[\dfrac{g(i)}{2}]arr[g(i)]=perm[ 
2
g(i)
​
 ],令 x = \dfrac{g(i)}{2}x= 
2
g(i)
​
 ,则该等式转换为 \textit{arr}[2x] = \textit{perm}[x]arr[2x]=perm[x],此时 x\in[0,\dfrac{n-1}{2}]x∈[0, 
2
n−1
​
 ];
如果 g(i)g(i) 为奇数,那么 \textit{arr}[g(i)] = \textit{perm}[\dfrac{n}{2} + \dfrac{g(i)-1}{2}]arr[g(i)]=perm[ 
2
n
​
 + 
2
g(i)−1
​
 ],令 x = \dfrac{n}{2} + \dfrac{g(i)-1}{2}x= 
2
n
​
 + 
2
g(i)−1
​
 ,则该等式转换为 \textit{arr}[2x-n-1] = \textit{perm}[x]arr[2x−n−1]=perm[x],此时 x \in[\dfrac{n+1}{2},\dfrac{n}{2}]x∈[ 
2
n+1
​
 , 
2
n
​
 ];
因此根据题目的转换规则可以得到以下对应关系:

当 0\le i < \dfrac{n}{2}0≤i< 
2
n
​
  时,此时 g(i) = 2ig(i)=2i;
当 \dfrac{n}{2} \le i < n 
2
n
​
 ≤i<n 时,此时 g(i) = 2i-(n-1)g(i)=2i−(n−1);
其中原排列中的第 00 和 n-1n−1 个元素的下标不会变换,我们无需进行考虑。 对于其余元素 i \in [1, n-1)i∈[1,n−1),上述两个等式可以转换为对 n-1n−1 取模,等式可以转换为 g(i) \equiv 2i \pmod {(n-1)}g(i)≡2i(mod(n−1))。

记 g^k(i)g 
k
 (i) 表示原排列 \textit{perm}perm 中第 ii 个元素经过 kk 次变换后的下标,即 g^2(i) = g(g(i)), g^3(i) = g(g(g(i)))g 
2
 (i)=g(g(i)),g 
3
 (i)=g(g(g(i))) 等等,那么我们可以推出:

g^k(i) \equiv 2^ki \pmod {(n-1)}
g 
k
 (i)≡2 
k
 i(mod(n−1))

为了让排列还原到初始值,原数组中索引为 ii 的元素经过多次变换后索引变回 ii,此时必须有:g^k(i) \equiv 2^ki \equiv i \pmod {(n-1)}g 
k
 (i)≡2 
k
 i≡i(mod(n−1))。我们只需要找到最小的 kk,使得上述等式对于 i\in[1,n-1)i∈[1,n−1) 均成立,此时的 kk 即为所求的最小变换次数。

当 i=1i=1 时,我们有

g^k(1) \equiv 2^k \equiv 1 \pmod {(n-1)}
g 
k
 (1)≡2 
k
 ≡1(mod(n−1))

如果存在 kk 满足上式,那么将上式两侧同乘 ii,得到 g^k(i) \equiv 2^ki \equiv i \pmod {(n-1)}g 
k
 (i)≡2 
k
 i≡i(mod(n−1)) 即对于 i \in [1, n-1)i∈[1,n−1) 恒成立。因此原题等价于寻找最小的 kk,使得 2^k \equiv 1 \pmod {(n-1)}2 
k
 ≡1(mod(n−1))。

由于 nn 为偶数,则 n-1n−1 是奇数,22 和 n-1n−1 互质,那么根据「欧拉定理」:

2^{\varphi(n-1)} \equiv 1 \pmod {(n-1)}
2 
φ(n−1)
 ≡1(mod(n−1))

即 k=\varphi(n-1)k=φ(n−1) 一定是一个解,其中 \varphiφ 为「欧拉函数」。因此,最小的 kk 一定小于等于 \varphi(n-1)φ(n−1),而欧拉函数 \varphi(n-1) \le n -1φ(n−1)≤n−1,因此可以知道 k \le n - 1k≤n−1 的,因此总的时间复杂度不超过 O(n)O(n)。

根据上述推论,我们直接模拟即找到最小的 kk 使得满足 2^k \equiv 1 \pmod {(n-1)}2 
k
 ≡1(mod(n−1)) 即可。

代码

Python3C++JavaC#CJavaScriptGolang

class Solution:
    def reinitializePermutation(self, n: int) -> int:
        if n == 2:
            return 1
        step, pow2 = 1, 2
        while pow2 != 1:
            step += 1
            pow2 = pow2 * 2 % (n - 1)
        return step
复杂度分析

时间复杂度:O(n)O(n),其中 nn 表示给定的元素。根据推论可以知道最多需要进行计算的次数不超过 nn,因此时间复杂度为 O(n)O(n)。

空间复杂度:O(1)O(1)。